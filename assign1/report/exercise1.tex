\documentclass[12pt]{article}
\usepackage[margin=1in]{geometry}
\usepackage{amsmath}
\usepackage{siunitx}
\usepackage{graphicx}
\usepackage{subcaption}
\usepackage{csvsimple}

\setlength{\parskip}{1em}

\begin{document}

\title{DD2424 Deep Learning in Data Science Assignment 1 Ex. 1}
\author{Lin Chun Hung, chlin3@kth.se}

\maketitle

In this exercise, a python version of multi-linear classifier was implemented.
The function to compute the gradient with analytical method was competed.

Unit tests for the function computing the gradient were written. In those unit tests,
I used \texttt{numpy.testing.assert\_allclose} to check if all the elements of two pairs of gradients,
\texttt{gradW} and \texttt{gradb}, computed from the analytical method and the numerical methods are closed.
In the assertion, the following equation is element-wise true, otherwise the unit test will fail.

\begin{equation*}
    % absolute(a - b) <= (atol + rtol * absolute(b))
    |a - b| \leq (\texttt{atol} + \texttt{rtol} * |b|)
\end{equation*}
where \texttt{atol} and \texttt{rtol} are the tolerance parameters.

Since the forward difference method is less accurate than the central difference method,
the tolerance parameters are different when comparing them to analytical method.
I set \texttt{atol} as \num{1e-7} and \texttt{rtol} as \num{1e-06} for the
forward difference method and set \texttt{atol} as \num{1e-9} and \texttt{rtol} as \num{1e-07}
for the central difference method.

We have four parameter sets and they are:
\begin{enumerate}
    \item  \texttt{lambda=0.0, n\_epochs=40, n\_batch=100, eta=0.1}
    \item  \texttt{lambda=0.0, n\_epochs=40, n\_batch=100, eta=0.01}
    \item  \texttt{lambda=0.1, n\_epochs=40, n\_batch=100, eta=0.01}
    \item  \texttt{lambda=1.0, n\_epochs=40, n\_batch=100, eta=0.01}
\end{enumerate}

% The cost function
\begin{figure}
    \centering
    \begin{subfigure}[b]{0.475\textwidth}
        \centering
        \includegraphics[width=\textwidth]{loss_case1.png}
        \caption[]%
        {{\small Parameter set 1}}
    \end{subfigure}
    \hfill
    \begin{subfigure}[b]{0.475\textwidth}
        \centering
        \includegraphics[width=\textwidth]{loss_case2.png}
        \caption[]%
        {{\small Parameter set 2}}
    \end{subfigure}
    \vskip\baselineskip
    \begin{subfigure}[b]{0.475\textwidth}
        \centering
        \includegraphics[width=\textwidth]{loss_case3.png}
        \caption[]%
        {{\small Parameter set 3}}
    \end{subfigure}
    \quad
    \begin{subfigure}[b]{0.475\textwidth}
        \centering
        \includegraphics[width=\textwidth]{loss_case4.png}
        \caption[]%
        {{\small Parameter set 4}}
    \end{subfigure}
    \caption[]
    {\small The graph of the training and validation loss computed after each epoch}
\end{figure}

% The learnt weight matrix
\begin{figure}
    \centering
    \begin{subfigure}[b]{0.475\textwidth}
        \centering
        \includegraphics[width=\textwidth]{wgt_case1.png}
        \caption[]%
        {{\small Parameter set 1}}
    \end{subfigure}
    \hfill
    \begin{subfigure}[b]{0.475\textwidth}
        \centering
        \includegraphics[width=\textwidth]{wgt_case2.png}
        \caption[]%
        {{\small Parameter set 2}}
    \end{subfigure}
    \vskip\baselineskip
    \begin{subfigure}[b]{0.475\textwidth}
        \centering
        \includegraphics[width=\textwidth]{wgt_case3.png}
        \caption[]%
        {{\small Parameter set 3}}
    \end{subfigure}
    \quad
    \begin{subfigure}[b]{0.475\textwidth}
        \centering
        \includegraphics[width=\textwidth]{wgt_case4.png}
        \caption[]%
        {{\small Parameter set 4}}
    \end{subfigure}
    \caption[]
    {\small The Images representing the learnt weight matrix after the completion of training}
\end{figure}

\begin{table}
    \centering
    \csvreader[tabular=|c|c|,respect percent=true,
        table head=\hline Parameter set & Accuracy \\\hline,
        late after line = \\\hline]
    {test_accuracy.csv}
    {param_set=\paramset,accuracy=\accuracy}
    {\paramset & \accuracy}
    \caption{<your caption>}
\end{table}

\end{document}
