\documentclass[12pt]{article}

\usepackage[margin=1in]{geometry}
\usepackage{amssymb}
\usepackage{amsmath}
\usepackage{graphicx}
\usepackage{subcaption}

\setlength{\parskip}{1em}


\newenvironment{question}[2][Question]{\begin{trivlist}
\kern10pt
\item[\hskip \labelsep {\bfseries #1}\hskip \labelsep {\bfseries #2.}]}{\end{trivlist}}


\begin{document}

\title{DD2424 Deep Learning in Data Science Assignment 3}
\author{Lin Chun Hung, chlin3@kth.se}

\maketitle

\section{Basic Part (Part 1)}
\begin{question}{i}
    I used the central difference method to calculate the numerical gradients
    with respect to all the network parameters and used it check against with the
    analytical gradients.

    I used limited dimension of data (say the input dimension is 10) and 10
    data to check with the numerical gradients.

    I checked a 2-layer network and an one batch norm layer network.
    For the pass criterion, I used the maximum relative error mentioned in assignment 1.
    The highest value of the maximum relative errors is 0.002. The layer giving 
    that value is the last linear layer. It is reasonably small and I considered
    it is okay as the non-linearity of the activation layers.

    I did the sanity check and the model can overfit with a batch of data.

    Therefore I think that the model is bug free.
\end{question}


\begin{question}{ii}
   
\end{question}

\end{document}
